\documentclass[xcolor=dvipsnames]{beamer}
\usepackage{caption}
\usepackage{subcaption}
\usepackage{dirtytalk}
 \usepackage[top=2cm, bottom=2cm, outer=0cm, inner=0cm]{geometry}
 
 \usepackage{wrapfig}
\useoutertheme{miniframes} % Alternatively: miniframes, infolines, split
%\useinnertheme{circles}
\definecolor{oldmauve}{rgb}{0.4, 0.19, 0.28}
\usecolortheme[named=oldmauve]{structure}
%\usecolortheme[named=Mahogany]{structure} % Sample dvipsnames color
\mode<presentation>{
     
     
%\usetheme{default}
 %\usetheme{AnnArbor}
 %\usetheme{Antibes}
 % \usetheme{Bergen}
%\usetheme{Berkeley}
 %\usetheme{Berlin}
 %\usecolortheme{beaver}
%\usetheme{Boadilla}
%\usetheme{CambridgeUS}
% \usetheme{Copenhagen}
%\usetheme{Darmstadt}
%\usetheme{Dresden}
 % \usetheme{Frankfurt}
 %\usetheme{Goettingen}
\usetheme{Hannover}
% \usetheme{Ilmenau}
\usetheme{JuanLesPins}
%\usetheme{Luebeck}
 %  \usetheme{Madrid}
%\usetheme{Malmoe}
%\usetheme{Marburg}
%\usetheme{Montpellier}
 % \usetheme{PaloAlto}
%\usetheme{Pittsburgh}
%\usetheme{Rochester}
%\usetheme{Singapore}
%\usetheme{Szeged}
%\usetheme{Warsaw}
\usepackage{tcolorbox}
\usepackage{lipsum}
\newcommand{\arctg}[1]{\text{arctg}(#1) }
\newcommand{\tg}[1]{\text{tg}(#1)}
}
\usepackage[utf8]{inputenc}
 \usepackage[demo]{graphicx}

\usepackage{pgfplots}
\pgfplotsset{/pgf/number format/use comma,compat=newest}
\usepackage{color}
\usepackage{amsmath,amsfonts,amssymb}
\usepackage{hyperref}
\usepackage{tikz}
\usepackage{enumerate}
\usepackage{listings}
\usepackage{graphicx}
\newtheorem{deff}{Definicija}
\newtheorem{thm}{\textrm{Teorem}}[section]
\newtheorem{defn}{\textrm{Definicija}}[section]
\newtheorem{pos}{\textrm{Posljedica}}[section]
\newtheorem{lema}{\textrm{Lema}}[section]
\newtheorem{pri}{\textrm{Primjer}}[section]
\newtheorem{prim}{\textrm{Primjedba}}[section]
\newenvironment{dokaz}{\noindent\textbf{\textrm{Dokaz:}}}{\rule{0.3cm}{0.3cm}}
\newenvironment{rje}{\noindent\textit{Rje\v senje:}}{$\clubsuit$}
\renewcommand{\dj}{d\kern-0.4em\char"16\kern-0.1em}
\renewcommand{\DJ}{\raise0.3ex\hbox{-}\kern-0.36em D}
\renewcommand{\figurename}{Slika}

 






\title{\bf LOGARITHMS {\vspace{.1cm}}\\   }
\author{ \bf 
}
\institute[] {\includegraphics[height=1.3in]{bbb.png} {{\vspace{.1cm}}\\ \bf  SCHOOL   {\vspace{.1cm}}\\ {\bf TUZLA 2020./21.}} } 
   


\begin{document}
\usebackgroundtemplate{%
\tikz\node[opacity=0.6,inner sep=0] {\includegraphics[height=\paperheight,width=\paperwidth]{kk.jpg}};}
\begin{frame}
 \maketitle 
\end{frame}

%\begin{frame}
%\frametitle{Sumário}
 %\tableofcontents
%\end{frame}


  
\section{ TYPES OF LOGARITHM}

\begin{frame}
\frametitle{ TYPES OF LOGARITHM  } 
\begin{block}{\textbf{The Fundamental Identity of Logarithms}}
 
    $$ a^{\log_{a} N} = N.$$ 
    
 $N$ is always positive and $a$ is always positive and different from 1. So, negative numbers and zero do not have  logarithms.
\end{block}
\begin{block}{\textbf{Common Logarithms}}
 
    Logarithms to the base $10$ are called \textbf{common logarithms}.\\
    We will use \textbf{$\log x$} to mean $\log_{10} x$.
    
 
\end{block}
\begin{block}{\textbf{Natural Logarithms}}
 
    Logarithms to the base $e$ are called \textbf{natural logarithms} or \textbf{Euler logarithms}  .\\
    We will use \textbf{$\ln x$} to mean $\log_{e} x$.
    
 
\end{block}
\end{frame}
\section{ PROPERTIES OF LOGARITHMS}

\begin{frame}
\frametitle{ PROPERTIES OF LOGARITHMS  } 
\begin{alertblock}{\textbf{Property 1}}
If the argument and the base of logarithm are equal, the logarithm is equal to 1. \\ Conversely, if the logarithm is 1, then the argument and the base are equal.

$$a = b \Longleftrightarrow \log_{a}b = 1 \hspace{0.5 cm}(a>0, \hspace{0.2 cm}a\ne 1 ) $$
   
\end{alertblock}
\begin{alertblock}{\textbf{Property 2}}
The logarithm of $1$ to any base is zero!
$$  \log_{a}1 = 0$$
\end{alertblock}
\end{frame}

\begin{frame}{PROPERTIES OF LOGARITHMS}
\begin{alertblock}{\textbf{Property 3}}
The logarithm of product of two  or more positive numbers to a given base is equal to the sum of the logarithms of the numbers to that base
$$  \log_{a}(x\cdot y) =\log_{a}x + \log_{a}y \hspace{0.5 cm} (x,\hspace{0.1 cm}y >0)$$
 
We can \textbf{generalize} this property as folllows:
$$ \log_{a}(x_{1} \cdot x_{2} \cdot x_{3}\cdot ... \cdot x_{k}) =\log_{a}x_{1} + \log_{a}x_{2}+...+  \log_{a}x_{k}.$$
\end{alertblock}

\end{frame}
\begin{frame}{PROPERTIES OF LOGARITHMS}
    \begin{alertblock}{\textbf{Property 3}}
The logarithm of a power of a positive number is equal to the product of the power and the logarithm of number
$$  \log_{a}(x^{m}) =m\cdot \log_{a}x \hspace{0.5 cm} m\in \mathbf{R},\hspace{0.1 cm} x>0   $$
 
\begin{center}
    
\textbf{Be careful!}
\end{center}
$$(\log_{a} x)^{m} \ne m\cdot \log_{a}x$$
\end{alertblock}
\end{frame}
\begin{frame}{PROPERTIES OF LOGARITHMS}
   \begin{alertblock}{\textbf{Property 4}}
The logarithm of the quotient of two positive numbers is equal to the difference between the logarithms of the dividend and the divisor to the same base:
$$  \log_{a}\left(\frac{x}{y}\right) = \log_{a}x - \log_{a}y   $$
\begin{center}
    
\textbf{Be careful!}
\end{center}
$$\frac{\log_{a} x}{\log_{a}y} \ne   \log_{a}x - \log_{a}y $$
 
 
 
\end{alertblock}
\end{frame}
\section{EXAMPLES}
\begin{frame}{EXAMPLES}
\begin{exampleblock}


\begin{enumerate}
    \item Calculate: $\log_{4} 2 + \log_{4} 8$
    \item Calculate: $\log_{2} 3 + \log_{2} 5 +    \log_{2} \frac{1}{15}  $ 
    \item Write each sum as a single logarithm:\begin{enumerate}
        \item [(a)] $(2\cdot \log_{3} a ) + (3\cdot \log_{3} b ) -  \log_{3} c $
        \item [(b)]  $(\frac{1}{2}\cdot \log_{2} a ) + (3\cdot \log_{2} b ) - \left( \frac{3}{2}\cdot\log_{3} c \right) $
    \end{enumerate} 
    \item Calculate: $\log_{2}  \sqrt[3]{2\cdot \sqrt{8 \cdot\sqrt[3]{16} }}$ 
     \item Write each logarithm as a sum or difference of logarithms to base $a$ logarithm:\begin{enumerate}
        \item [(a)] $  \log_{a} \frac{b^{3}c^{2}}{d^{4}e^{5}}$
        \item [(b)]  $  \log_{a} \frac{\sqrt[5]{(b+c)^{2}}}{ (d-e)^{3}}$
    \end{enumerate} 
\end{enumerate}
\end{exampleblock}
    
\end{frame}
\section{ CHANGING THE BASE OF LOGARITHM}

\begin{frame}
\frametitle{ CHANGING THE BASE OF LOGARITHM }
\begin{block}{\textbf{Property 6} }
Raising the base of a logarithm to a non zero power is the same as dividing the logarithm by that power

$$  \log_{a^{n}}x = \frac{1}{n} \cdot \log_{a}x   $$
   
 \end{block}
 \begin{exampleblock}

\begin{enumerate}
    \item Write the following expression as a single logarithm to base 3.\\
    $\log_{\frac{1}{3}}7 + 2 \log_{9}49  -\log_{\sqrt{3}}\frac{1}{7}$
   
\end{enumerate}
   \end{exampleblock}
\end{frame}
\begin{frame}{CHANGING THE BASE OF LOGARITHM}
    \begin{alertblock}{Note}
    $$  \log_{a^{n}}x^{m} = \frac{m}{n} \cdot \log_{a}x   $$
    $$  \log_{a }x  =    \log_{a^{n}}x^{n}  $$
    \end{alertblock}
     \begin{exampleblock}

\begin{enumerate}
    \item Evaluate the expressions.\\
    $\log_{36}\frac{1}{6} +  \log_{\frac{1}{5}}\frac{1}{125}  +\log_{8}128 +  \log_{\frac{1}{3}}9 $
   
\end{enumerate}
   \end{exampleblock}
  
\end{frame}
\begin{frame}{CHANGING THE BASE OF LOGARITHM}
   \begin{block}{\textbf{Property 7 - \textbf{The Change Base Formula}} }
Let $a,$ $b$, and $x$ be positive numbers such that $a\ne 1$ and $b\ne 1$, then:


$$  \log_{a}x = \frac{   \log_{b}x}{ \log_{b}a}   $$
   
 \end{block}
  \begin{block}{ Conclusion }
We can easily derive the following properties:
 


$$  \log_{a}b = \frac{1}{ \log_{b}a}   $$
$$  \log_{a}b \cdot  \log_{b}a =1   $$
   
 \end{block}
\end{frame} 
\begin{frame}{Examples}
\begin{exampleblock}

\begin{enumerate}
    \item Evaluate:
    $ \frac{4}{\log_{2}12} + \frac{2}{\log_{3}12}  $
   \item Calculate $\log_{20}200 $ in terms of $p$ if $\log_{5}2 = p$
   \item $\log_{a}b = c$ is given. Express each logarithm in terms of $c$:
   \begin{enumerate}
       \item [(a)] $\log_{a^{2}b}ab^{2} $
       \item [(b)] $\log_{\frac{a^{3}}{b^{2}}}ab^{4}$
   \end{enumerate}
   \item Calculate each logarithm in terms of $a$, using the relation given:
   \begin{enumerate}
       \item [(a)] $\log 25 $; $a = \log 2$
       \item [(b)] $\log_{3}18$; $a = \log_{3}12$
       \item [(c)] $\log_{12}27$; $a = \log_{6}16$
   \end{enumerate}
   
\end{enumerate}
\item Simplify: $\sqrt{25^{\frac{1}{\log_{6}5}} + 49^{\frac{1}{\log_{8}7}} }$
\item Show that $\log_{2}3 \cdot \log_{5}7 \cdot \log_{11}13= \log_{2}13 \cdot \log_{5}3 \cdot \log_{11}7$
   \end{exampleblock}
\end{frame}

\begin{frame}{ } 
    \begin{block}{  Property 8   }
 
$b^{ \log_{a}c}  = c^{ \log_{a}b}  $ \\ for $a,b,c >0$ and $a\ne 1.$

   
 \end{block}
 \begin{exampleblock}{$\star$}
 Calculate:
 $$2^{ \log_{3}5} - 5^{ \log_{3}2}  $$
 \end{exampleblock} 
\end{frame}

\end{document}


